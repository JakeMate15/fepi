\section{Introducción}

En el presente documento se abordan los conceptos básicos relacionados con proyectos, con énfasis en el ámbito informático. Se definen términos como proyecto, proyecto informático, sus elementos fundamentales, así como los procesos de formulación y evaluación. Finalmente, se describe el ciclo de vida que atraviesa un proyecto desde su inicio hasta su cierre.

\section{Proyecto}

Un \textbf{proyecto} es un conjunto de actividades \textbf{planificadas y coordinadas} que se realizan para lograr un objetivo específico. Todos los proyectos requieren una planificación previa y suelen tener \textbf{límites} definidos, como un tiempo determinado para su ejecución, un \textbf{presupuesto} y recursos disponibles \cite{concepto}. En resumen, un proyecto busca \textbf{alcanzar resultados concretos} dentro de condiciones establecidas (tiempo, recursos y alcance).

\section{Proyecto Informático}

Un \textbf{proyecto informático} es un tipo de proyecto enfocado en el área de la \textbf{tecnología de la información}. Esto significa que involucra la creación o mejora de sistemas computacionales o software. En un proyecto informático participan \textbf{recursos tecnológicos} (como hardware y software) y personas especializadas, todos \textbf{coordinados} para obtener un resultado deseado en un sistema de información \cite{disenowebakus}. Al igual que cualquier proyecto, tiene un inicio y un final definidos, con objetivos claros en el ámbito tecnológico.

\section{Elementos de un Proyecto}

Los proyectos comparten una serie de \textbf{elementos fundamentales} que se deben considerar al planificarlos \cite{concepto}. Algunos de los principales son:

\begin{itemize}
    \item \textbf{Objetivo o finalidad:} lo que se pretende lograr o el problema que se quiere resolver con el proyecto.
    \item \textbf{Alcance:} definición de las \textbf{entregas} o resultados que incluirá el proyecto (qué se hará y qué no se hará).
    \item \textbf{Recursos:} todo lo necesario para llevar a cabo el proyecto, incluyendo recursos \textbf{humanos} (personas o equipo de trabajo), \textbf{materiales} y \textbf{financieros} (presupuesto).
    \item \textbf{Tiempo o cronograma:} el período de duración del proyecto y la planificación de sus etapas o tareas en el tiempo.
    \item \textbf{Restricciones:} limitaciones o condiciones que afectan al proyecto, como restricciones de presupuesto, de tiempo, de alcance u otras.
\end{itemize}

Estos elementos ayudan a \textbf{organizar} y entender un proyecto, asegurando que esté bien definido antes de comenzar.

\section{Evaluación de Proyecto}

La \textbf{evaluación de un proyecto} es el proceso de \textbf{analizar y comprobar} los resultados del proyecto en función de sus objetivos. Puede ocurrir al final del proyecto, para verificar si se cumplió el objetivo general en el tiempo y con el éxito esperado \cite{concepto}. También puede realizarse antes o durante el proyecto (evaluación inicial o seguimiento) para determinar su \textbf{viabilidad} o para introducir mejoras. En términos sencillos, evaluar un proyecto significa revisar qué tan bien funcionó o funcionará, identificando \textbf{aciertos} y \textbf{aspectos por mejorar}.

\section{Formulación de Proyectos}

La \textbf{formulación de proyectos} es la etapa en la que se \textbf{diseña y planifica detalladamente} un proyecto \textbf{antes} de llevarlo a cabo. En este proceso se recopila y organiza toda la información necesaria para estructurar el proyecto. Esto incluye definir claramente la \textbf{idea inicial}, los \textbf{objetivos}, las actividades requeridas, los recursos necesarios y los posibles resultados esperados. En otras palabras, la formulación consiste en \textbf{elaborar un plan completo} que sirva de guía para ejecutar el proyecto de forma viable \cite{evaluacionproyectos}. Una buena formulación ayuda a prevenir problemas y asegurar que todos entiendan cómo se desarrollará el proyecto.

\section{Ciclo de Vida de un Proyecto}

El \textbf{ciclo de vida de un proyecto} se refiere a las \textbf{etapas o fases} que atraviesa un proyecto desde que inicia hasta que termina. Típicamente, un proyecto pasa por cuatro fases principales \cite{concepto}:

\begin{enumerate}
    \item \textbf{Inicio o diagnóstico:} se define la \textbf{idea del proyecto}, su objetivo general y la necesidad que busca atender. Aquí se identifica el problema a resolver o la oportunidad a aprovechar.
    \item \textbf{Planificación o diseño:} se detalla \textbf{cómo} se llevará a cabo el proyecto. Se planifican las tareas, se asignan \textbf{recursos}, se establecen cronogramas y se determina el presupuesto. Esta fase asegura que todo esté organizado antes de ejecutar.
    \item \textbf{Ejecución:} es la puesta en marcha del plan. Se realizan las \textbf{actividades} programadas y se construyen los entregables del proyecto. Durante esta etapa también se lleva un control para verificar que todo avance según lo previsto.
    \item \textbf{Cierre o evaluación:} al finalizar, se \textbf{evalúan} los resultados obtenidos comparándolos con los objetivos iniciales. Se comprueba si el proyecto cumplió su propósito y se documentan las \textbf{lecciones aprendidas}. En esta fase el proyecto se da por terminado.
\end{enumerate}

Cada etapa del ciclo de vida es importante para asegurar el éxito del proyecto, ya que permite ordenar el trabajo de forma lógica y gestionar eficientemente el tiempo y los recursos.

\section{Conclusión}

Comprender los conceptos básicos de un proyecto es fundamental para poder formularlo y evaluarlo correctamente. Conocer sus elementos, su ciclo de vida y las diferencias entre formulación y evaluación permite planificar de manera ordenada y aumentar las probabilidades de éxito en cualquier proyecto informático.
